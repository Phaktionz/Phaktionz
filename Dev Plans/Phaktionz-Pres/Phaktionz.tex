\documentclass{beamer}
\usepackage[utf8]{inputenc}
\usepackage{graphicx}
%Beamer theme
\usetheme{Madrid}
% Setting Title Page----------------------
\usecolortheme{beaver}
% Font -----------------------------------
\usefonttheme{structuresmallcapsserif}

\title[Phaktionz Official Rules] %optional
{Phaktionz Official Rules}

\subtitle{We Arise the Battlefield of Realms}

\author[Phaktionz Rules Committee] % (optional, for multiple authors)
{Phaktionz Rules Committee}

\institute[] % (optional)
{\inst{Casual Card Cafe}}

\date[] % (optional)
%------------------------------------------------------------
%The next block of commands puts the table of contents at the
%beginning of each section and highlights the current section:

\AtBeginSection[]
{
  \begin{frame}
    \frametitle{Table of Contents}
    \tableofcontents[currentsection]
  \end{frame}
}
%------------------------------------------------------------

\begin{document}
\frame{\titlepage}


%------------------------------------------------------------
\section{Phaktionz Basic Rules}
\begin{frame}
    \frametitle{Basic Idea}
    \textrm{Goal: The goal of Phaktionz is to deck out your opponent. The way this is done is the idea of taking damage 
    by sending the top card of your deck to the deck out pile. \\Whoever reaches 0 loses. This game has two different 
    types of cards, Summons and Invocations, Summons are the units you play with and Invocations may be mystical powers
    or tactical advantage cards that are used to advance in the game.}
\end{frame}


\begin{frame}
    \frametitle{Layout}
    \begin{figure}
        \includegraphics[scale=0.22]{images/field.png}
        \caption{\textrm{The Layout of the Battlefield}}
    \end{figure}
\end{frame}

\begin{frame}
    \frametitle{Explaining Layout}
\begin{itemize}
    \item Deck: The deck must be 50 cards and will be placed in the designated spot 
    \item Deck-Out Pile: This is where summons that are demoted are put, or when taken damage. 
    \item Tier 2/1: The spots that have Tier 2/1 is where Tier 2/1 are able to be placed unless an exception a Tier 3 cannot be placed in these spots. 
    \item Tier 3: This spot can only have Tier 3s be placed on it. 
    \item Realm Invocation: A Realm Invocation is used to bring battlefield advantage, and can be placed on its designated spot. 
\end{itemize}
    

\end{frame}



\begin{frame}
    \frametitle{Phases}
    %
    \textrm{While playing Phaktionz there are 5 phases during a turn, the phases go as follows:} 
    %
    \begin{alertblock}{Draw Phase}
        \textrm{The Player begins their turn by a drawing a card from their deck}
    \end{alertblock}
    %
    \begin{alertblock}{Main Phase}
       \textrm{The Player can now place their summons or activate their invocations}
    \end{alertblock}
    %
    \begin{alertblock}{Combat Phase}
        \textrm{The Player can now battle different summons on the opposing battlefield}
    \end{alertblock}
\end{frame}
%
\begin{frame}
    \frametitle{Phases Pt 2}
    \begin{alertblock}{Final Cast}
        \textrm{The Player can only play any last invocations they would like to play in their turn}
    \end{alertblock}
    %
    \begin{alertblock}{End Phase}
        \textrm{The Player now ends their turn so the opponent can start theirs}
    \end{alertblock}
\end{frame}




%--------------------------------------------------
\section{Single Player vs Multi-Player}





%--------------------------------------------------
\section{Formats}
\end{document}
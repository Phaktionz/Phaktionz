\documentclass[12pt, letterpaper]{article}
\title{Phaktionz Rules}
\author{Casual Card Cafe}
\date{September-December 2020}
\usepackage{color}
\usepackage{hyperref}
\hypersetup{
colorlinks=true, % make the links colored
linkcolor=black, % color TOC links in blue
urlcolor=red, % color URLs in red
linktoc=all % 'all' will create links for everything in the TOC
}
\usepackage{graphicx}
\graphicspath{ {./} }
\begin{document}
\maketitle
\pagenumbering{arabic}
\newpage
\tableofcontents
\newpage
  \section{Rules}
  \begin{itemize}
      \item To win you must deck out your opponent, or in other words bring their deck to 0 cards, where
            no more cards can be drawn.
     \item Your deck must contain 50 cards
     \item You must start with 5 cards in your hand
     \item You are only able to play 4 copies of a card unless limits are placed upon it
     \item The Phases of a turn is as follows:
     \begin{enumerate}
         \item Draw Phase (Player draws a card from their deck)
         \item Main Phase (Can place summons or cast invocations)
         \item Combat Phase (Battle using your summons)
         \item Final Cast (Can cast invocations)
         \item End Phase (Ends turn)
     \end{enumerate}
  \end{itemize}
  \begin{figure}
  \begin{center}
     \includegraphics[scale=0.25]{field.png}
     \caption{Battlefield format} 
  \end{center}
   \end{figure}
   \newpage
   \section{Battlefield and Game Mechanics}
   \paragraph{Game Mechanics: \\}
   \begin{itemize}
       \item When a summon battle it becomes disabled (turned sideways)
       \item To place a tier 2 or higher summon, you must demote tiers total to the summon's tier. 
       For example, a tier 2 may be placed by demoting a tier 2 or 2 tier 1s.
       \item At the start of the game, after the turn order is chosen, both players may mulligan any cards in their hand once.
        \item If a card’s ability were to break one of these rules, the cards ability takes precedence.
   \end{itemize}
   \paragraph{Terms:\\}
   \begin{itemize}
       \item Summons: Units that battle in the battlefield
       \item Invocations: Sorcery that may be casted to gain benefit by contering, helping summons, etc.
       \item Abled: The position in which a unit may battle
       \item Disabled: The position in which a unit is unable to battle (This is done with your summon being sideways)
       \item Demote: To have a summon leave the battlefield
       \item Exile: To remove from play a summon
       \item Tiers: Represents the rank of a summon, tier 1 being the lowest and 3 the highest
       \item DMG: The amount of cards a summon can deal an opponent to lose, this is indicated on the card, and when battles an opposing summon, is dealt the difference.
   \end{itemize}
   \paragraph{Conditions:}
   \begin{itemize}
       \item L/x: Limit x per turn
       \item Lx: Limit x per match
       \item Cx: Choose x summons 
       \item Dx: Demote x summons  
   \end{itemize}
   \section{Types}
   \paragraph{\textbf{There are two type of summons: }}
   \begin{enumerate}
       \item \textbf{Striker:} can battle any opposing summons but not directly
       \item \textbf{Tech:} can only battle opposing summons in the same column, but if there are no opposing summons in their column they can battle directly.
   \end{enumerate}
   \paragraph{\textbf{There are 4 type of invocations: }}
   \begin{enumerate}
       \item \textbf{Regular:} These can only be used on your turn, and are used for your benefit 
       \item \textbf{Counter:} These can be used at anytime with the right condition, these are used to stop an opponent's action
       \item \textbf{Weapon:} These can be attatched underneath a summon, and is used to enhance the summon
       \item \textbf{Realm:} These stay on the battlefield, and affect only your side of the battlefield 
   \end{enumerate}
   %\paragraph{Abilities:}
   %\begin{itemize}
       %\item Mummify: demote a summon and remove from play a summon in the end zone and create a mummy {Vanilla/Dmg 3/Striker/Tier2}. 
       %\item Bury: demote a summon/token of yours, and if so exile a summon of your opponent's of the same tier on the battlefield.
       %\item Curse: Choose 1 summon on both battlefields, they cannot go into abled position as long as they both remain on the battlefield, at the end phase both players take a damage. 
       %\item Immortality: For the rest of the turn, the summon cannot be demoted from battle. 
       %\item Switch: Return a disabled summon to your hand and place a summon of the same tier to where the disabled summon was in abled position.  
   %\end{itemize}
\end{document}